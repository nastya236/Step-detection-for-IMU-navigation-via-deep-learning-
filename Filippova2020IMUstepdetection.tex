\documentclass[12pt, twoside]{article}
\usepackage{jmlda}
\newcommand{\hdir}{.}
\usepackage[utf8]{inputenc}
\usepackage[english,russian]{babel}
\usepackage{graphicx}
\usepackage{hyperref}       % hyperlinks
\usepackage{url}            % simple URL typesetting
\usepackage{booktabs}       % professional-quality tables
\usepackage{amsfonts}       % blackboard math symbols
\usepackage{nicefrac}       % compact symbols for 1/2, etc.
\usepackage{microtype}
\usepackage{lipsum}
\usepackage{longtable}
\usepackage{graphicx}
\usepackage{subcaption}
\usepackage{float}
\usepackage{amsmath}
\usepackage{multirow}

\begin{document}

\title
    [Step detection via deep learning] % краткое название; не нужно, если полное название влезает в~колонтитул
    {Step detection via deep learning}
\author
    [А.\,В.~Филиппова] % список авторов (не более трех) для колонтитула; не нужен, если основной список влезает в колонтитул
    {А.\,В.~Филиппова, Т.~Гадаев, В.\,В.~Стрижов} % основной список авторов, выводимый в оглавление
    [А.\,В.~Филиппова$^1$, Т.~Гадаев$^1$, В.\,В.~Стрижов$^{1}$] % список авторов, выводимый в заголовок; не нужен, если он не отличается от основного
% \email
   % {islamov.ri@phystech.edu; grabovoy.av@phystech.edu;  strijov@ccas.ru}
%\thanks
%    {Работа выполнена при
%     %частичной
%     финансовой поддержке РФФИ, проекты \No\ \No 00-00-00000 и 00-00-00001.}
\organization
    {$^1$Московский физико-технический институт}
\abstract
    {В данной работе рассматривается задача предсказания траектории человека по показаниям аксселрометра и гироскопа, которые установлены в телефоне. Так как система отсчета, связанная с устройством, постоянно вращается и движется ускоренно относительно мировой системы, поставленная задача не является тривиальной. Существует много различных необучаемых алгоритмов для описания траектории человека. Минус этих алгоритмов в том, что модель не может подстраиваться под конкретную постановку задачи и учитывать детали (пол, возраст, особенность походки объекта). В данной работе предлагается нейросетевой подход для решения задачи, а также описываются полезные эвристики. 
    
    Наше исследование можно разделить на три логические части: проектирование устойчивых к ошибкам гироскопов кватернионов, прогнозирование изменения положения на фиксированном периоде с помощью различных нейронных сетей и применение идеи детекции шагов с целью улучшения показаний модели и уменьшения дрифта.
\bigskip
\noindent


\textbf{Ключевые слова}: \emph {аксселерометр и гироскоп; детекция шагов ;кватернионы; дрифт; нейросетевой подход; предсказание траектории}
}

\maketitle
\linenumbers

\section{Введение}

Задача точного определения положения смартфона в пространстве,и, как следствие, оценка местоположения объекта решается с высокой точностью на открытых площадках с использованием GPS ~\cite{mohamed1999adaptive}. Современные технологии демонстрируют отличные результаты при отклонении менее чем на несколько метров ~\cite{rahiman2013overview}. Однако у системы есть недостаток: она требует открытого пространства между устройством и спутником для передачи радиосигналов. В реальном мире нас часто окружают деревья, неровности ландшафта, высокие здания. Качество геолокации снижается из-за отражения радиоволн. Например, использование GPS для отслеживания траекторий внутри зданий практически бесполезно ~\cite{dedes2005indoor}. В этом случае используются методы, основанные на данных других датчиков. Наиболее распространенными датчиками IMU смартфона являются гироскоп, магнитометр и акселерометр. Основной проблемой такого подхода является накопление ошибок позиционирования из-за дрейфа, вызванного несовершенствами и шумом в датчиках (тут будет ссылка на статью моего куратора, которая пока не была опубликована). В данной работе предлагается по мимо методов, описанных в статье  (тут будет ссылка на статью моего куратора, которая пока не была опубликована) использовать детекцию шагов с целью улучшения показаний модели и уменьшения дрифта.

\section{Постановка задачи}

Задача состоит в том, чтобы найти суперпозицию функций, которые мы обозначим как $F_{\text{tr+st}}$, которая преобразует данные датчиков в оценку траектории, , которая будет близка к истиной, а также дает оценку вероятности совершения шага вдоль траетории в каждый момент времени.
\begin{equation}
    \label{eq:general_ps}
    \argmin\limits_{F_{\text{tr+st}}}\mathcal{L}\left(F_{\text{tr+st}}\left(\cal{A}, \cal{W}\right), \cal{T}, \cal{S}\right)
\end{equation}{}
\newpage
В качестве функции потерь предлагается использовать комбинированную функцию  $\mathcal{L}\left(F_{\text{tr}}\left(\cal{A}, \cal{W}\right), \cal{T}, \cal{S}\right) = \textbf{MSE}\left(F_{\text{tr}}\left(\cal{A}, \cal{W}\right), \cal{T}\right) + \textbf{BCElogloss}\left(F_{\text{st}}\left(\cal{A}, \cal{W}\right), \cal{S}\right)$.

Данная функция потерь позволяет обучить модель таким образом, чтобы для вещественных выходов модели решалась задача регрессии, для категориальных - классификации. 

Для оценки качества предсказаний используются следующие показатели:  \textbf{RMSE}~\eqref{eq:metric_rmse}, (\textbf{MIE})~\eqref{eq:metric_mie}, (\textbf{GAP})~\eqref{eq:metric_gap}.



\bibliographystyle{unsrt}
\bibliography{References}

\end{document}

